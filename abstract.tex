\chapter*{Abstract} \label{chapter:abstract}
\addcontentsline{toc}{chapter}{\nameref{chapter:abstract}}

Materials and their properties play a vital role in most applications we use on 
a daily basis. Many of the revolutions in industry are instigated by the 
discovery, by accident or design, of a new material that makes an application 
commercially viable. Historically, the study of materials has largely relied on 
an intuition-driven trial and error approach. However, considering the enormous 
design space of possible materials, as well as the fact that many of the 
straightforward materials have already been discovered, this process has become 
too expensive and time-consuming. 

Since the middle of the 20\textsuperscript{th} century, a new paradigm in 
materials science has been developing, where computer simulations are used to 
calculate the properties of materials from first principles. This new approach 
has steadily become more and more successful, pushed forward by the 
ever-increasing performance of modern computers and rapid progress in 
theoretical methods. During the past few decades, computational materials 
science has started evolving more and more into a predictive tool instead of 
simply offering theoretical insight into the physical processes of materials 
of interest. In combination with increasingly available tools for automating 
the required calculations, this has lead to the concept of \textit{in silico} 
materials design, where large numbers of compounds are investigated using 
computer simulations in order to gauge their potential for a specific application.

Among the most successful theoretical frameworks for computational materials 
science is density functional theory, which can determine the electronic 
structure of many compounds with ever increasing accuracy using a reasonable 
amount of computational resources. However, the connection between the 
electronic structure of a material and the property of interest for a specific 
application is rarely trivial. The main goal of this thesis is to provide or 
improve this connection, by analyzing existing metrics for flaws or anomalies, 
and developing new descriptors of material properties as well as the tools for 
calculating them using automated workflows. These methods are then applied to 
a set of topics, including solar cells, Li-ion batteries and ion-induced 
secondary electron emission. The structure of the thesis is as follows:

\begin{enumerate}[]

\vfill
\item \textbf{Chapter~\ref{chapter:intro}} briefly introduces the concept of 
\textit{in silico} materials design, and provides a guide to the reader of 
this thesis for navigating and consulting the available resources.

\vfill
\item \textbf{Chapter~\ref{chapter:dft}} explains the density functional theory 
framework, as well as some practical computational techniques for calculating 
the electronic structure using this framework.

\vfill
\item \textbf{Chapter~\ref{chapter:automation}} outlines the workflows used for 
the automation of the required density functional theory calculations of each 
descriptor or metric.

\vfill
\item \textbf{Chapter~\ref{chapter:slme}} discusses the Shockley-Queisser limit 
and spectroscopic limited maximum efficiency, two metrics used to determine the 
potential of a material as the absorber layer of a single-junction solar cell. 
Next, it makes a comparison of the CuAu-like and chalcopyrite phase in the 
context of thin-film photovoltaics.

\vfill
\item \textbf{Chapter~\ref{chapter:batteries}} presents an investigation of the 
stability of the oxygen framework of Li-rich \ce{Li2MnO3} and \ce{Li2IrO3} 
battery cathodes, as well as a limited substitution of \ce{Mn} as a potential 
recipe for improving the structural stability of these materials. Moreover, it 
discusses the energy landscapes of \ce{LiCB11H12} and \ce{NaCB11H12} polyborane 
salts in the context of solid electrolytes.

\vfill
\item \textbf{Chapter~\ref{chapter:quotas}} discloses a new model for 
calculating the secondary electron emission yield from ions neutralized at a 
semiconductor and metal surface, and applies this descriptor to a set of 
elemental surfaces spanning the periodic table in a high-throughput approach.

\end{enumerate}

\afterpage{\null\newpage}
