\chapter*{Abstract (Nederlands)} \label{chapter:abstract_nederlands}
\addcontentsline{toc}{chapter}{\nameref{chapter:abstract_nederlands}}

Materialen en hun eigenschappen spelen een cruciale rol in vele toepassingen die we dagelijks gebruiken. Veel van de revoluties in de industrie worden veroorzaakt door de ontdekking, per ongeluk of ontwerp, van een nieuw materiaal. De conventionele studie van materialen is grotendeels gebaseerd op een intu\"itiegestuurde \textit{trial and error} aanpak, maar gezien de enorme ontwerpruimte van mogelijke materialen is dit proces te duur en tijdrovend.

Sinds het midden van de 20\textsuperscript{e} eeuw is een nieuw paradigma in de materiaalwetenschap aan het opkomen, waarbij computersimulaties worden gebruikt om de eigenschappen van materialen \textit{ab initio} te berekenen. In combinatie met de steeds toenemende performantie van computers en de snelle vooruitgang in theoretische methoden, biedt computationeel materiaalonderzoek meer en meer een betrouwbare manier om de eigenschappen van materialen te voorspellen. Dit heeft geleid tot het concept van \textit{in silico} materiaalontwerp, waarbij een groot aantal materialen wordt onderzocht met behulp van computersimulaties om hun potentieel voor een specifieke toepassing na te gaan.

Een van de meest succesvolle theoretische kaders voor computationeel materiaalonderzoek is dichtheidsfunctionaaltheorie, waarmee de elektronische structuur van veel materialen met steeds toenemende nauwkeurigheid kan bepaald worden. Het verband tussen de elektronische structuur van een materiaal en de eigenschap die ons interesseert voor een specifieke toepassing is echter zelden triviaal. Het hoofddoel van dit proefschrift is om deze verbinding te bieden of te verbeteren, door bestaande metrieken voor fouten of afwijkingen te analyseren en nieuwe descriptoren van materiaaleigenschappen te ontwikkelen. Daarnaast worden deze methoden toegepast op een reeks onderwerpen, waaronder zonnecellen, Li-ion batterijen en secundaire elektronenemissie door neutralisatie van een ion aan een oppervlak. De structuur van het proefschrift is als volgt:

\begin{enumerate}[]

\vfill
\item \textbf{Hoofdstuk~\ref{chapter:intro}} introduceert kort het concept van \textit{in silico} materiaalontwerp.
\vfill
\item \textbf{Hoofdstuk~\ref{chapter:dft}} legt dichtheidsfunctionaaltheorie uit, evenals enkele praktische computationele technieken voor het berekenen van de elektronische structuur binnen dit kader.
\vfill
\item \textbf{Hoofdstuk~\ref{chapter:automation}} schetst de workflows die worden gebruikt voor de automatisering van de vereiste \textit{ab initio} berekeningen van elke descriptor of metriek.
\vfill
\item \textbf{Hoofdstuk~\ref{chapter:slme}} bespreekt de Shockley-Queisser limiet en de \textit{spectroscopic limited maximum efficiency}, twee metrieken die worden gebruikt om het potentieel van een materiaal als de absorberende laag van een zonnecel te bepalen. Vervolgens maakt het een vergelijking van de \textit{CuAu-like} en chalcopyrietfase in de context van dunne-film fotovoltaïsche cellen.
\vfill
\item \textbf{Hoofdstuk~\ref{chapter:batteries}} presenteert een onderzoek naar de stabiliteit van het zuurstofkader van Li-rijke \ce{Li2MnO3}- en \ce{Li2IrO3}-batterijkathoden, evenals een beperkte substitutie van Mn als een potentieel recept voor het verbeteren van de structurele stabiliteit van deze materialen. Bovendien bespreekt het de energielandschappen van \ce{LiCB11H12} en \ce{NaCB11H12} polyboraanzouten in de context van vastestofelektrolyten.
\vfill
\item \textbf{Hoofdstuk~\ref{chapter:quotas}} beschrijft een nieuw model voor het berekenen van de secundaire elektronenemissie van ionen geneutraliseerd op een halfgeleider- en metaaloppervlak, en past deze toe op een reeks elementaire oppervlakken in een \textit{high-throughput} benadering.

\end{enumerate}
