\section*{Abstract (Nederlands)} \label{chapter:abstract_nederlands}
\addcontentsline{toc}{chapter}{\nameref{chapter:abstract_nederlands}}

Materialen en hun eigenschappen spelen een cruciale rol in de meeste toepassingen die we dagelijks gebruiken. Veel van de revoluties in de industrie worden veroorzaakt door de ontdekking, per ongeluk of ontwerp, van een nieuw materiaal dat een toepassing commercieel levensvatbaar maakt. Historisch gezien is de studie van materialen grotendeels gebaseerd op een intuïtiegestuurde trial and error-benadering. Gezien de enorme ontwerpruimte van mogelijke materialen en het feit dat veel van de eenvoudige materialen al zijn ontdekt, is dit proces te duur en tijdrovend geworden.

Sinds het midden van de 20e eeuw is een nieuw paradigma in de materiaalwetenschap ontwikkeld, waarbij computersimulaties worden gebruikt om de eigenschappen van materialen uit eerste principes te berekenen. Deze nieuwe aanpak is steeds succesvoller geworden, vooruitgeschoven door de steeds toenemende prestaties van moderne computers en de snelle vooruitgang in theoretische methoden. In de afgelopen decennia is computational materials science meer en meer begonnen te evolueren naar een voorspellende tool in plaats van alleen theoretisch inzicht te bieden in de fysieke processen van materialen van interesse. In combinatie met steeds meer beschikbare tools voor het automatiseren van de vereiste berekeningen, heeft dit geleid tot het concept van silico-materiaalontwerp, waarbij een groot aantal verbindingen wordt onderzocht met behulp van computersimulaties om hun potentieel voor een specifieke toepassing te peilen.

Een van de meest succesvolle theoretische kaders voor computationele materiaalwetenschap is dichtheidfunctionaaltheorie, die de elektronische structuur van veel verbindingen met steeds toenemende nauwkeurigheid kan bepalen met behulp van een redelijke hoeveelheid computationele bronnen. Het verband tussen de elektronische structuur van een materiaal en de eigenschap van interesse voor een specifieke toepassing is echter zelden triviaal. Het hoofddoel van dit proefschrift is om deze verbinding te bieden of te verbeteren, door bestaande metrics voor fouten of afwijkingen te analyseren en nieuwe descriptoren van materiaaleigenschappen te ontwikkelen, evenals de hulpmiddelen voor het berekenen ervan met behulp van geautomatiseerde workflows. Deze methoden worden vervolgens toegepast op een reeks onderwerpen, waaronder zonnecellen, Li-ionbatterijen en door ionen veroorzaakte secundaire elektronenemissie. De structuur van het proefschrift is als volgt:

\begin{enumerate}[]

\vfill
\item \textbf{Hoofdstuk~\ref{chapter:intro}} introduceert in het kort het concept van \textit{in silico} materialenontwerp en biedt een gids voor de lezer van dit proefschrift voor het navigeren en raadplegen van de beschikbare bronnen.


\vfill
\item \textbf{Hoofdstuk~\ref{chapter:dft}} legt het raamwerk van de functionele dichtheidstheorie uit, evenals enkele praktische computationele technieken voor het berekenen van de elektronische structuur met behulp van dit raamwerk.

\vfill
\item \textbf{Hoofdstuk~\ref{chapter:automation}} schetst de workflows die worden gebruikt voor de automatisering van de vereiste dichtheidstheorieberekeningen van elke descriptor of metriek.

\vfill
\item \textbf{Hoofdstuk~\ref{chapter:slme}} bespreekt de Shockley-Queisser-limiet en de spectroscopische beperkte maximale efficiëntie, twee metrieken die worden gebruikt om het potentieel van een materiaal als de absorberende laag van een single-junction zonnecel te bepalen. Vervolgens maakt het een vergelijking van de CuAu-achtige en chalcopyrietfase in de context van dunne-film fotovoltaïsche cellen.

\vfill
\item \textbf{Hoofdstuk~\ref{chapter:batteries}} presenteert een onderzoek naar de stabiliteit van het zuurstofraamwerk van Li-rijke \ce{Li2MnO3}- en \ce{Li2IrO3}-batterijkathoden, evenals een beperkte vervanging van Mn als een potentieel recept voor het verbeteren van de structurele stabiliteit van deze materialen. Bovendien bespreekt het de energielandschappen van \ce{LiCB11H12} en \ce{NaCB11H12} polyboraanzouten in de context van vaste elektrolyten.
% \vfill
% \item \textbf{Hoofdstuk~\ref{chapter:quotas}} beschrijft een nieuw model voor het berekenen van de secundaire elektronenemissieopbrengst van ionen geneutraliseerd op een halfgeleider- en metaaloppervlak, en past deze descriptor toe op een set elementaire oppervlakken die het periodiek systeem omspannen in een high-throughput-benadering.

\end{enumerate}

\afterpage{\null\newpage}
