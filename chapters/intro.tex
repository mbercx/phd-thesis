\begin{refsection} 
\pagestyle{chapter} 

\chapter{In Silico Materials Design} \label{chapter:intro}

\setlength{\epigraphwidth}{4in} 
\epigraph{\textit{``If you can look into the seeds of time, \\And say which 
grain will grow and which will not,\\Speak then to me, who neither beg nor 
fear,\\Your favors nor your hate."}}{Banquo - \textit{Macbeth Act I, Scene 
iii}} 
\vspace{3em} 

The discovery of new materials has driven many of the greatest technological 
revolutions in human history. In ancient times, the advent of the use of bronze 
and iron was an advancement that now marks an age. The invention of modern 
concrete changed the rules of civil engineering. More recently, the discovery 
of bakelite, the first truly synthetic material, led to the onset of the modern 
plastics industry. Clearly, material innovation is an essential component of 
technological development.

Currently, we know the properties of less than 1\% of all materials, and it can 
require decades to identify new compounds for a technological 
application~\cite{Jain2013}. This process needs to be more efficient, by reducing 
both the time and cost of material development. Over the past few decades, there 
has been a significant increase in the capability of high-performance computing, 
which shows no signs of abating. Computer simulations that use sophisticated 
quantum models have the power to analyze many properties of materials, offering 
a relatively cheap and effective method to discover new potential candidates for 
any application. Recently developed computational high-throughput methods provide 
an automated procedure to screen large amounts of compounds~\cite{Curtarolo2013, 
Jain2015}. The combination of these rapidly evolving techniques with a suitable 
descriptor or selection metric has the potential of identifying the most 
promising materials from the many different possible structures.

In this chapter, I aim to provide a brief introduction to the concept of \textit{in 
silico} materials design, as well as frame my research within this broader context. 
Section~\ref{intro:sec-in_silico} starts by explaining the term \textit{in silico}. 
Next, I combine this with the second part of the title of this chapter and connect 
it to the chapters of my thesis in Section~\ref{intro:sec-materials_design}. 
Finally, Section~\ref{intro:sec-guide} discusses some conventions used throughout 
the thesis to help the reader navigate the document.

\newpage
\section{In Silico} \label{intro:sec-in_silico}

Although the theoretical modeling of materials has a long history, the practise 
of describing physical relations using equations largely started during the 
scientific revolution in the 16\textsuperscript{th} and 17\textsuperscript{th} 
century. With the advent of modern computers, theoretical modeling was taken 
to the next level by facilitating the solution of these equations for more 
complex systems. This allowed the introduction of ``computer experiments", i.e. 
simulations, whose results can be analyzed and compared with those from 
experiment. \textit{In silico} refers to the use of such computer simulations, 
usually on a large computational infrastructure called a supercomputer, to 
solve intricate questions in a range of scientific disciplines. 

For many problems in solid state physics, the equation that we aim to solve 
\textit{in silico} is the Schr\"odinger equation. More specifically, we want 
to determine the electronic structure of a material, as many important 
properties can be derived from an understanding of the electronic states and 
their response to e.g. electromagnetic fields. Our method of choice for 
calculating the electronic structure \textit{ab initio}, i.e. from 
first principles, is density functional theory (\gls{DFT}), which is treated in 
Chapter~\ref{chapter:dft}. Although there are other methods available, \gls{DFT} 
strikes a good balance between accuracy and computational workload, allowing 
for the study of a large number of systems.

\section{Materials Design} \label{intro:sec-materials_design}

Traditionally, many materials have been discovered experimentally either by 
chance or trial and error, largely driven by intuition. Such a trial and error 
approach can either be performed one material at a time, or in a so-called 
high-throughput fashion. A popular experimental example of the latter is how 
Edison tested over 6000 materials to serve as a filament for the incandescent 
lightbulb~\cite{E.1918}. Once a material is found that has the right properties, 
it is investigated in order to understand its structure and its relation to the 
origin of these properties. This approach, often referred to as direct design, 
has been the predominant source of new materials over the past centuries.

However, a potentially more effective approach to discover new materials is 
inverse design~\cite{Zunger2018}, where instead of finding materials and studying 
their properties, we start with a certain application in mind and then look for 
materials that exhibit the right properties for said application. There are 
different strategies to execute this approach, such as: high-throughput 
computational efforts that generate large databases~\cite{Curtarolo2013, Jain2011}, 
that rely on evolutionary techniques to generate new and improved 
structures~\cite{Oganov2018} or to use machine learning and data mining to discover 
trends and patterns~\cite{Draxl2019}. The combination of these ideas with the 
development of ever more accurate computational techniques has put us on the brink of a revolution in material science, where large scale automated calculations can guide us to the next generation of materials for commercial applications.

When using \gls{DFT} as our method of choice for calculating the properties of the materials under investigation, the relation between the fundamental quantities we can obtain from \gls{DFT} and the property of interest is often complex. As phrased by Gerbrand Ceder, ``there is no quantum operator for a better car"~\cite{Ceder2010}. In order to obtain insight from the \gls{DFT} results, we need domain-specific expertise to develop descriptors or metrics for the application of interest. During my PhD I have worked on a diverse set of topics, but the main goal has always been to use results from \textit{ab initio} calculations to study materials with a certain application in mind. Moreover, I have developed workflows -- discussed in Chapter~\ref{chapter:automation} -- for automating the \gls{DFT} calculations and open-source Python packages for performing the subsequent analysis, i.e. transfering the \gls{DFT} results to the task of materials design. 

Here is a brief overview of how the various topics I have worked on can be framed in the context of \textit{in silico} materials design:

\begin{itemize}[]

\item \textbf{Chapter~\ref{chapter:slme}} concerns an application analysis of the spectroscopic limited maximum efficiency (\gls{SLME}), a metric developed to judge the potential of materials for solar cells. Here, the combination of the workflow described in Section~\ref{automation:sec-optics} with the post-processing tools for calculating the \gls{SLME} allows for an automated high-throughput screening of materials for photovoltaic applications.

\item \textbf{Chapter~\ref{chapter:batteries}} concerns an investigation of Li-rich layered oxides, a class of cathode materials that have demonstrated large energy densities, as well as the energy landscapes of \ce{LiCB11H12} and \ce{NaCB11H12} polyborane salts in the context of solid electrolytes. The computation of the properties of interest here is more involved, and hence the workflows described in Sections~\ref{automation:sec-configurations}, \ref{automation:sec-dimer} and \ref{automation:sec-landscape} are necessary to make this process feasible for even a limited number of materials\footnote{One of the Master's students I helped to supervise calculated that it would have taken him 1200 hours to manually perform and process the calculations for his thesis, which largely focused on a single material.}.

\item \textbf{Chapter~\ref{chapter:quotas}} concerns the calculation of the secondary electron emission from slow ions neutralized at a surface. Here we have both developed a model for calculating this descriptor from the \gls{DFT} results, as well as a workflow (Sec.~\ref{automation:sec-surface}) for automating its calculation.

\end{itemize}

\section{Guide to this thesis} \label{intro:sec-guide}

One of the main objectives of this thesis is to provide the reader with the necessary information and tools to be able to reproduce my work, as well as extend their application to other materials. To achieve this, the workflows I designed are documented in Chapter~\ref{chapter:automation}, along with a description of the parts that they consist of. All of the underlying code is also freely available on github, and the necessary steps to set up the calculations and process the data are detailed in Jupyter notebooks. I have also added a more conventional description of the computational settings to Appendix~\ref{appendix:sec-results}, in order to gather all of these details in one place and not unnecessarily interupt the discussion of the results on the various topics.

To make this document easier to navigate, I have added a large amount of links to the text. These are consistently colored as follows: \link{}{dark red text links to other parts of the thesis}, whereas \href{https://github.com/mbercx/phd-thesis}{blue text links to a web page}, which will be automatically opened in a tab of your default browser. Besides these links, each section that contains results also has two symbols next to its header:

\vspace{1em}
\begin{minipage}{0.09\textwidth}
\includegraphics[width=3em]{figures/jupyter.png}
\end{minipage}
\begin{minipage}{0.9\textwidth}
Links to the corresponding Jupyter notebooks that have been used to set up the calculations and process the data for the results presented in the section. Several results can also be explored interactively, depending on the topic. For most sections, the figures also have been set up in a Jupyter notebook.
\end{minipage}

\vspace{1em}
\begin{minipage}{0.09\textwidth}
\includegraphics[width=2.5em]{figures/silicon3.png}
\end{minipage}
\begin{minipage}{0.9\textwidth}
Links to a section in Appendix~\ref{appendix:sec-results} which contains a more conventional description of the computational parameters used for the calculation of the properties presented in the section.
\end{minipage}

\vspace{1em}
I highly recommend the reader to look up the key of their PDF reader of choice for returning to the previous page, i.e. before clicking on one of the links that navigates to a different section of the text. For Apple's preview, this is \code{Cmd+[}. For Acrobat reader, this should be \code{Alt+Leftarrow}. Although the content of the Jupyter notebooks might not be entirely clear to someone who is not familiar with \textit{ab initio} calculations, I still invite everyone to at least browse some of the notebooks corresponding to the chapter he or she is most interested in. Moreover, there are several interactive notebooks for exploring the results, which can be run online and require very little experience from the user. In the end though, this information is most valuable for other researchers who are studying similar topics as the ones presented in my thesis, and can hopefully learn from the resources I provide.

\clearpage
\pagestyle{biblio}
\printbibliography

\end{refsection} 
