\begin{refsection}

\chapter{Vienna Ab initio Simulation Package} \label{appendix:sec-VASP}

In order to solve the many-body problem using DFT, we need a software package that is able to implement the theory numerically on a computer cluster. Currently, there is a wide selection of such packages available to computational scientists, each with their respective advantages and disadvantages.

The software we have used to obtain the results presented in this thesis is the Vienna Ab initio Simulation Package (VASP)\cite{vasp}. This software package is particularly suited for materials science, using a plane wave basis set and offering PAW datasets for each atomic species. It can calculate an approximate solution to the many-body Schr\"odinger equation within the DFT formalism or the HF approximation, including the possibility of mixing to utilize hybrid functionals. VASP also employs a set of efficient iterative procedures to find the ground state of a system, and allows parallelization of the calculations on multi-core machines.

In this section we present a concise overview of the different files used by VASP, discussing their purpose, and take a closer look at the input parameters and their relation to the theory. We also present some of the choices for the parameters which were used to produce the results presented in the following chapters.

\section{Files}

We begin by taking a brief look at the input files. In order to start a calculation using VASP, we must have at least the following files:

\begin{itemize}

\phantomsection \label{appendix:sec-INCAR}
\item \texttt{INCAR}: Contains the input parameters for the calculation. Various settings can be adjusted according to the needs of the user through a large amount of \textit{tags}, see Section~\ref{sec:input}. This can be considered the most important input file, in the sense that it has the most content and therefore significantly affects the results of the simulation. Because of this, it is also more prone to be the cause of errors. 

\phantomsection \label{appendix:sec-POSCAR}
\item \texttt{POSCAR}: Specifies the lattice vectors as well as the atomic positions of the system. The user is free to specify the atomic positions in cartesian or direct coordinates. In order to make sure there are no mistakes in the POSCAR file, it is important to first visualize the structure. 

\phantomsection \label{appendix:sec-KPOINTS}
\item \texttt{KPOINTS}: Defines the $\mathbf{k}$ point mesh (Sec.~\ref{dft:sec-kpoints}), either by explicitly entering all the points or using an automated Monkhorst-Pack grid. In the case of band structure calculations, it is also possible for the user to enter certain symmetry lines along which to calculate the band structure of the crystal.

\phantomsection \label{appendix:sec-POTCAR}
\item \texttt{POTCAR}: Concatenation of the PAW datasets for the different atoms present in the crystal structure. VASP supplies various types of POTCAR files, corresponding to different methods of PAW construction. It is important to make sure the ordering of the atoms is the same for the POTCAR and POSCAR file.

\phantomsection \label{appendix:sec-STOPCAR}
\item \texttt{STOPCAR}:

\end{itemize}


During the calculation, VASP produces a set of output files from which the user can extract the data necessary for his or her research. Here's a (non-exhaustive) list of the VASP output files:

\begin{itemize}

\phantomsection \label{appendix:sec-OUTCAR}
\item \texttt{OUTCAR}: The general output of the calculation is printed in the this file, whose verbosity is determined by a tag in the \vasp{INCAR} file.

\phantomsection \label{appendix:sec-CONTCAR}
\item \texttt{CONTCAR}: For relaxation calculations, the final lattice vectors and atom positions can be found in this file.

\end{itemize}

The CHGCAR and WAVECAR file contain the charge density and wave functions of the electrons, and are both rather large compared to the other VASP files. These files can also be used as input files for continuation jobs. The OSZICAR file presents an overview of the total energy at each SCF iteration, as well as some other properties, for each ionic step in the case of a relaxation calculation. The set of irreducible $\mathbf{k}$-points, along with their respective weights, can be found in the IBZKPT file. The Density Of States (DOS) and integrated DOS can be plotted from the DOSCAR file, whereas the EIGENVAL file contains the Kohn-Sham eigenvalues for all $\mathbf{k}$-points, which can be used to plot the band structure.

\section{Input Parameters} \label{sec:input}

To end the chapter, we take a look at some of the input tags which are set by the user to determine the attributes of the calculation. This is by no means an exhaustive list; we simply focus on a selection of tags that were especially relevant for the results presented in this thesis. For the complete list, we refer the reader to the VASP manual~\cite{vasp}.

\begin{itemize}

\phantomsection \label{appendix:sec-ENCUT}
\item \texttt{ENCUT}: 

\phantomsection \label{appendix:sec-PREC}
\item \texttt{PREC}: 

\phantomsection \label{appendix:sec-ALGO}
\item \texttt{ALGO}: 

\phantomsection \label{appendix:sec-EDIFF}
\item \texttt{EDIFF}: 

\phantomsection \label{appendix:sec-ISMEAR}
\item \texttt{ISMEAR}: 

\phantomsection \label{appendix:sec-SIGMA}
\item \texttt{SIGMA}: 

\phantomsection \label{appendix:sec-ISIF}
\item \texttt{ISIF}: 

\phantomsection \label{appendix:sec-LOPTICS}
\item \texttt{LOPTICS}: 

\phantomsection \label{appendix:sec-NEDOS}
\item \texttt{NEDOS}: 

\phantomsection \label{appendix:sec-ISPIN}
\item \texttt{ISPIN}: 

\phantomsection \label{appendix:sec-MAGMOM}
\item \texttt{MAGMOM}: 

\phantomsection \label{appendix:sec-NBANDS}
\item \texttt{NBANDS}: 

\phantomsection \label{appendix:sec-ISIF}
\item \texttt{ISIF}: 

\phantomsection \label{appendix:sec-CSHIFT}
\item \texttt{CSHIFT}: 

\item \textbf{ENCUT} and \textbf{PREC} influence the accuracy of the calculation. ENCUT is equal to the $E_{cut}$ value described in Section~\ref{sec:planewave}, and specifies the size of the plane wave basis set.  PREC mainly defines the spacing of several grids that govern the precision of the simulation, such as the fast Fourier transform~\cite{numrec} grid.

\item \textbf{ISMEAR} specifies the \textit{smearing} method and \textbf{SIGMA} the smearing width. Smearing is a technique that replaces the step function in the integral of the filled bands:
\begin{equation}\label{eq:intsmear}
\sum_n \frac{1}{\Omega_{BZ}}\int_{BZ} \epsilon_{n\mathbf{k}} \Theta(\epsilon_{n\mathbf{k}} - E_F) d\mathbf{k},
\end{equation}
by a smooth function $f(\{\epsilon_{n\mathbf{k}}\})$. The main advantage of using smearing methods is that the integral in Eq.~(\ref{eq:intsmear}) can be calculated accurately using a relatively sparse $\mathbf{k}$-mesh (Sec.~\ref{sec:kpoints}). The main smearing method used for the results presented in this thesis is \textit{Gaussian} smearing~\cite{gaussian} (ISMEAR = 0). Another method used to solve the integral in Eq.~(\ref{eq:intsmear}) is the \textit{tetrahedron} method~\cite{tetrahedron} (ISMEAR = -5), which linearly interpolates $\epsilon_{n\mathbf{k}}$ between two \textbf{k}-points.


\item We can choose the relaxation algorithm using the \textbf{IBRION} tag, with the degrees of freedom specified by \textbf{ISIF}. The latter also determines whether the stress tensor is calculated or not. To determine the lattice structure of the compounds presented in this work, we used the conjugate gradient algorithm~\cite{conjgrad}~\cite{iteratieve}(IBRION = 2). Another important tag for the relaxation calculation is \textbf{EDIFFG}, which sets the value for the convergence condition of the relaxation procedure.

\item When we want to use hybrid functionals, \textbf{LHFCALC} must be set to 'TRUE'. In this case, \textbf{AEXX} determines the mixing parameter, whereas \textbf{HFSCREEN} specifies the separation $\omega$ (Sec.~\ref{sec:funct}). 

\end{itemize}

\chapter{Plasmons} \label{sec:appendix-plasmon}

\section{Surface Plasmons}

Hagstrum's Auger transform can be written as a convolution:
\begin{equation}
\begin{aligned}
T\left[\frac{\varepsilon + \varepsilon_0 - E_i}{2}\right] &= \int_{\Delta\varepsilon_v}\int_{\Delta\varepsilon_v} D_v (\varepsilon_1) D_v (\varepsilon_2) \delta(\varepsilon - \varepsilon_1 - \varepsilon_2 + \varepsilon_0 - E_i) d\varepsilon_1 d\varepsilon_2. \\
&= \int_{\Delta\varepsilon_v} D_v (\varepsilon_1) D_v (\varepsilon + \varepsilon_0 - E_i - \varepsilon_1) d\varepsilon_1 \\ 
&= (D_v * D_v) (\varepsilon + \varepsilon_0 - E_i)
\end{aligned}
\end{equation}
which can also be written as:
\begin{equation}
T\left[\frac{\varepsilon + \varepsilon_0 - E_i}{2}\right] = D_v(\varepsilon) * D_v(\varepsilon + \varepsilon_0 - E_i)
\end{equation}

Here, $E_{n}(\varepsilon) = D_v(\varepsilon + \varepsilon_0 - E_i)$ corresponds to the distribution of energies released by the neutralization of the incoming ion\footnote{This should be more clear when we substitute e.g. $\varepsilon = 0$ in $E_{n}(\varepsilon)$. We then obtain $D_v(\varepsilon_0 - E_i)$, i.e. the number of electrons in the valence band that are at an equal depth (in relation to the vacuum level) as the lowest unoccupied state of the incoming ion. Clearly, electrons at this level do not pass energy through any Auger process, as this is a resonant neutralization. Another way of understanding this is that an electron at energy level $\varepsilon_0 - E_i + \varepsilon$ releases an energy of $\varepsilon$ when it neutralizes the incoming ion.}.  In order to include surface plasmon excitations, we consider the probability that a plasmon with energy $E_{sp}$ is excited instead of an Auger process:
\begin{equation}
P_{sp}(E_{sp}) = \frac{ \frac{A}{\lambda_{aug}} \text{Im}\left[-\frac{1}{\epsilon(E_{sp}) + 1}\right]}{\frac{A}{\lambda_{aug}} \text{Im}\left[-\frac{1}{\epsilon(E_{sp}) + 1}\right] + 1}
\end{equation}
As the surface plasmon excitation process is considered to be resonant, $E_{sp}$ must correspond to the energy released in the neutralization of the incoming ion, i.e. $\varepsilon$ in $E_{n}(\varepsilon)$. The leftover distribution of energies passed to other valence electrons through an Auger process is:
\begin{equation}
    E_{aug}(\varepsilon) = E_{n}(\varepsilon) (1 - P_{sp}(\varepsilon))
\end{equation}
We substitute this distribution in the expression for the Auger transform:
\begin{equation}
\begin{aligned}
T`\left[\frac{\varepsilon + \varepsilon_0 - E_i}{2}\right] &= D_v(\varepsilon) * E_{aug}(\varepsilon) \\ &= D_v(\varepsilon) * (D_v(\varepsilon + \varepsilon_0 - E_i) (1 - P_{sp}(\varepsilon)))
\end{aligned}
\end{equation}
And use this Auger transform instead of the original one in the nominator of Eq.~(5.4). Note that the normalization in the denominator should be the same as before, else the number of Auger electron due to the neutralization will still be normalized to one per incoming ion.

\section{Volume Plasmons}

Consider the distribution of excited electrons $N_i(\varepsilon)$. The probability that an excited electron at energy $\varepsilon$ induces a volume plasmon of energy $E_{vp} < \varepsilon - \varepsilon_F$ is given by:
\begin{equation}
P_{vp}(\varepsilon, E_{vp}) dE_{vp} = \frac{L(E_{vp})dE_{vp}}{\int_0^{\varepsilon - \varepsilon_F} L(E) dE }\left[1-e^{\left(\int_0^{\varepsilon - \varepsilon_F} L(E) dE\right) A t_e }\right]
\end{equation}
The fraction of electrons which excite a volume plasmon is then equal to:
\begin{equation}
    N_{vp}^{-}(\varepsilon) = \int_0^\infty N_i(\varepsilon) P_{vp}(\varepsilon, E_{vp}) dE_{vp}.
\end{equation}
These are substracted from the excited electron density. However, these electrons are not lost, they have simply lost an energy equal to the volume plasmon energy $E_{vp}$. Hence, we have to add the following density to the excited density again:
\begin{equation}
    N_{vp}^{+}(\varepsilon) = \int_0^\infty N_i(\varepsilon + E_{vp}) P_{vp}(\varepsilon, E_{vp}) dE_{vp}.
\end{equation}
The story doesn't end there, however. Volume plasmons do not couple with transversal waves [CITE], and hence cannot decay radiatively. Most likely, the plasmon will decay as a single electron excitation. In order to model this, we first calculate the distribution of plasmons with an energy $E_{vp}$:
\begin{equation}
    D_{vp} (E_{vp})= \int_0^\infty N_i(\varepsilon) P_{vp}(\varepsilon, E_{vp}) d\varepsilon
\end{equation}
When these plasmons decay, they release an energy equal to $E_vp$. In other words, this distribution should be convoluted with the density of the occupied states $D_v(\varepsilon)$ in order to obtain the density of excited electrons due to volume plasmon decay:
\begin{eqnarray}
    N_{vp}^{d}(\varepsilon) &=& \frac{1}{n_{vp}}(D_v * D_{vp}) (\varepsilon) \\
    &=& \frac{1}{n_{vp}} \int_{\Delta\varepsilon_v} D_v(\varepsilon_1) D_{vp}(\varepsilon - \varepsilon_1) d\varepsilon_1 \\
    &=& \frac{1}{n_{vp}} \int_{\Delta\varepsilon_v} D_v(\varepsilon_1) 
    \int_0^\infty N_i(\varepsilon_2) P_{vp}(\varepsilon_2, \varepsilon - \varepsilon_1) d\varepsilon_2 d\varepsilon_1 
\end{eqnarray}
where $n_{vp}$ is a normalization factor. Of course, this distribution should be normalized to the number of excited plasmons, i.e.:
\begin{equation}
    n_{vp} = \int_0^\infty N_{vp}^{-}(\varepsilon) d\varepsilon
\end{equation}

\section{Poisson Process} \label{sec:appendix-poisson}

\chapter{Computational Details} \label{appendix:sec-computational}

All calculations are performed in the density functional theory (DFT) framework, as implemented in the Vienna Ab initio Simulation Package (VASP) [13,14]. The projector augmented wave (PAW) method [15] was used to make a distinction between the core and valence electrons, with the standard VASP recommended choice for the number of valence electrons. 

\section{Results} \label{appendix:sec-results}

\subsection{Solar Cells} \label{appendix:sec-solar}

We make a selection of ten compounds for which we can compare the calculated efficiency of the CA phase with the CH results of Yu and Zunger~\cite{Yu2012}. The CA and CH structure are studied using a first-principles approach within the Density Functional Theory (DFT) formalism, as implemented in the Vienna Ab initio Simulation Package~\cite{Kresse1993, Kresse1996, Kresse1996b} (VASP). The Projector Augmented Wave (PAW) method~\cite{Blchl1994} is applied, and the electrons that are treated as valence electrons are underlined in Table~\ref{tab:valElec}. The exchange-correlation functional is calculated using the Generalized Gradient Approximation (GGA) of Perdew-Burke-Ernzerhof (PBE)~\cite{Perdew1996}. The energy cutoff for the plane wave basis is set to 350 eV, and a 4$\times$4$\times$4 Monkhorst-Pack~\cite{Monkhorst1976} (MP) mesh is used for sampling the first Brillouin zone. Electronic convergence is obtained when the energy difference between two electronic steps is smaller than $10^{-4}$~\si{\electronvolt}. The structure is considered converged when the forces on the atoms are all below $10^{-2}$~\si{\electronvolt}/\si{\angstrom}. \\\vspace{0.5em}
Because an accurate band gap is important for the correct evaluation of the efficiency, we perform single shot G$_0$W$_0$~\cite{Hybertsen1985} calculations on top of HSE06~\cite{Heyd2006}. However, in order to accurately update the quasiparticle energies within the G$_0$W$_0$ approximation, it is necessary to consider the semi-core electrons as valence electrons within the PAW framework~\cite{Fuchs2007}. Hence, we treat the 3$s$, 3$p$ and 3$d$ (4$s$, 4$p$ and 4$d$) orbitals as valence states for the Ga (In) atoms for the G$_0$W$_0$@HSE06 calculations of the band gap. In addition, we use a well converged 8$\times$8$\times$8 MP mesh, an increased energy cutoff of 400~\si{\electronvolt} and a large amount of unoccupied bands (600 in total). \\\vspace{0.5em}
The optical properties are calculated within the Random Phase Approximation (RPA), using the long wavelength expression for the imaginary part of the dielectric tensor~\cite{Gajdos2006, Harl2007}. The real part of the dielectric tensor is determined using the Kramers-Kronig relation\footnote{The Kramers-Kronig relation is calculated by VASP using a complex shift (``CSHIFT''). After calculating the real part, however, VASP also recalculates the corresponding imaginary part. Since the complex shift introduces a broadening, this causes an earlier onset of the imaginary part, and consequently in the absorption coefficient. In order to prevent this, we commented out the line in the VASP code that recalculates the imaginary part.}. In order to get an accurate description of the energy levels, the exchange-correlation energy is calculated with the HSE06 functional, which has been reported~\cite{Wan2013} to produce optical properties close to those obtained from experiment for CuIn(S$_x$Se$_{x-1}$)$_2$. We found that it is enough to sample the Brillouin zone using a 12$\times$12$\times$12 MP mesh to obtain a converged dielectric tensor. The number of unoccupied bands is increased to at least three times the number of occupied bands. Because of the tetragonal symmetry of the CA structure, the resulting dielectric tensor is diagonal and has two independent components $\varepsilon_{xx}$ (=$\varepsilon_{yy}$) and $\varepsilon_{zz}$. Since we make no assumptions about the direction from which the photons enter the absorber layer, we average the diagonal components to derive the dielectric function $\varepsilon (E) = \varepsilon^{(1)} (E) + i \varepsilon^{(2)} (E) $ at energy $E$. Finally, in order to obtain a more accurate onset of the absorption spectrum, we shift the imaginary part of the dielectric function to the G$_0$W$_0$@HSE06 band gap, and recalculate the real part using the Kramers-Kronig relations. 

\begin{table}[htbp]
\centering
\setlength{\captionmargin}{20pt}
\renewcommand{\arraystretch}{1.2}
\caption{\label{tab:valElec}Electron configuration of the atoms.}
\begin{tabular}{c@{\hskip 2 em}l}\hline
Element & Configuration \\\hline
Cu & [Ar] \underline{3d$^{10}$4s$^1$} \\
Ag &[Kr] \underline{4d$^{10}$5s$^1$} \\
Ga &[Ar] 3d$^{10}$\underline{4s$^2$4p$^1$}\\
In &[Kr] 4d$^{10}$\underline{5s$^{2}$4p$^1$} \\
S &[Ne] \underline{3s$^2$3p$^4$} \\
Se &[Ar]  3d$^{10}$\underline{4s$^{2}$4p$^4$} \\
Te &[Kr] 4d$^{10}$\underline{5s$^2$5p$^4$}\\
\hline
\end{tabular}
\end{table}

\subsection{Li-ion Batteries}

The wave functions of the valence electrons are expanded in a plane wave basis set, using a high energy cutoff equal to 500 eV, which is advisable for structures containing oxygen. The bulk structures were optimized with the hybrid HSE06 functional [16]. A suitably converged Monkhorst-Pack mesh [17] was chosen for the k-point sampling of the Brillouin zone, with a k-mesh spacing smaller than 0.05 \AA. Geometry optimizations were performed with a Gaussian smearing of 0.05 eV for semiconductor and insulators, and a Methfessel-Paxton [18] smearing of 0.2 eV for metallic structures, followed by a self consistent field calculation using the tetrahedron method [19], for an accurate calculation of the total energies. The convergence criterion on the electronic optimization is set at 10" eV, and 10"' eV for the geometric optimization. As Ir is known to exhibit a strong spin-orbit interaction, non-collinear calculations including spin-orbit coupling were performed for calculating the magnetic moment and density of states of LixIrO3. The dimer reaction energy and kinetics of both compounds were calculated in a 2x2x2 supercell, where we switched to the PBE+U functional [20,21] in order to make the dimer screening computationally feasible. A range of choices for the U parameter were tested to closely match the magnetic moments and lattice constants of the HSE06 calculation for the bulk structures, and we settled on 3.9 eV for Mn and 4.0 eV for Ir. Activation energies were calculated using the nudged elastic band (NEB) method [22]. 

\begin{table}[h]
\centering
\caption{Overzicht invloed U-waarden voor O1-\ce{Li_{0.5}MnO_3}}
  \begin{tabular}{rrrrrr}
    \toprule
    U (eV) & Mn-3d ($\mu_B$) & O-2p ($\mu_B$) & $|\vec{a}|$ (\si{\angstrom}) & $|\vec{c}|$ (\si{\angstrom}) & $\Delta E$ dimer (meV) \\
    \midrule
    1.0 & 2.319 & -0.248 & 9.779 & 8.483 &  \\ 
    2.0 & 2.660 & -0.348 & 9.886 & 8.250 & +153 \\
    3.0 & 2.874 & -0.408 & 9.955 & 8.214 & +27 \\ 
    3.5 & 2.977 & -0.438 & 9.978 & 8.223 & +6 \\ 
    4.0 & 3.068 & -0.463 & 10.000 & 8.231 & -70 \\ 
    4.5 & 3.159 & -0.489 & 10.033 & 8.217 & -138 \\ 
    5.0 & 3.252 & -0.515 & 10.068 & 8.216 & -181 \\  
    6.0 & 3.444 & -0.570 & 10.147 & 8.204 & -343 \\ 
   \bottomrule
   \end{tabular}
\end{table}

\subsubsection{Structure and Li-configurations} \label{appendix:sec-structure}
All calculations are performed using the \textbf{PBE+U functional  functional}, where a U-correction of 3.9 eV is applied to the $3d$ orbitals of Mn. The wave functions of the valence electrons are expanded in a plane wave basis set, using a high \textbf{energy cutoff equal to 500 eV}, which is advisable for structures containing oxygen. A suitably converged Monkhorst-Pack mesh was chosen for the k-point sampling of the Brillouin zone, with a k-mesh spacing smaller than 0.05 \AA. Geometry optimizations were performed with a \textbf{Gaussian smearing of 0.05 eV} for semiconductor and insulators, and a Methfessel-Paxton smearing of 0.2 eV for metallic structures, followed by a self consistent field calculation using the tetrahedron method, for an accurate calculation of the total energies. The convergence criterion on the electronic optimization is set at $10^{-6}$~eV, and $10^{-3}$ eV for the geometric optimization.

\subsubsection{Oxidation} \label{appendix:sec-oxidation}
All calculations were performed with the \textbf{hybrid HSE06 functional} [16].The wave functions of the valence electrons are expanded in a plane wave basis set, using a high \textbf{energy cutoff equal to 500 eV}, which is advisable for structures containing oxygen. A suitably converged Monkhorst-Pack mesh [17] was chosen for the k-point sampling of the Brillouin zone, with a k-mesh spacing smaller than 0.05 \AA. Geometry optimizations were performed with a \textbf{Gaussian smearing of 0.05 eV} for semiconductor and insulators, and a Methfessel-Paxton [18] smearing of 0.2 eV for metallic structures, followed by a self consistent field calculation using the tetrahedron method [19], for an accurate calculation of the total energies. The convergence criterion on the electronic optimization is set at $10^{-6}$~eV, and $10^{-3}$ eV for the geometric optimization. As Ir is known to exhibit a strong spin-orbit interaction, non-collinear calculations including spin-orbit coupling were performed for calculating the magnetic moment and density of states of LixIrO3. The dimer reaction energy and kinetics of both compounds were calculated in a 2x2x2 supercell, where we switched to the PBE+U functional [20,21] in order to make the dimer screening computationally feasible. A range of choices for the U parameter were tested to closely match the magnetic moments and lattice constants of the HSE06 calculation for the bulk structures, and we settled on 3.9 eV for Mn and 4.0 eV for Ir. Activation energies were calculated using the nudged elastic band (NEB) method [22]. 

\subsubsection{Dimer} \label{appendix:sec-dimer}
All calculations were performed with the PBE+U functional [20,21]. The wave functions of the valence electrons are expanded in a plane wave basis set, using a high \textbf{energy cutoff equal to 500 eV}, which is advisable for structures containing oxygen. A suitably converged Monkhorst-Pack mesh [17] was chosen for the k-point sampling of the Brillouin zone, with a k-mesh spacing smaller than 0.05 \AA. Geometry optimizations were performed with a \textbf{Gaussian smearing of 0.05 eV} for semiconductor and insulators, and a Methfessel-Paxton [18] smearing of 0.2 eV for metallic structures, followed by a self consistent field calculation using the tetrahedron method [19], for an accurate calculation of the total energies. The convergence criterion on the electronic optimization is set at $10^{-6}$~eV, and $10^{-3}$ eV for the geometric optimization. As Ir is known to exhibit a strong spin-orbit interaction, non-collinear calculations including spin-orbit coupling were performed for calculating the magnetic moment and density of states of LixIrO3. The dimer reaction energy and kinetics of both compounds were calculated in a 2x2x2 supercell, where we switched to  A range of choices for the U parameter were tested to closely match the magnetic moments and lattice constants of the HSE06 calculation for the bulk structures, and we settled on 3.9 eV for Mn and 4.0 eV for Ir. Activation energies were calculated using the nudged elastic band (NEB) method [22]. 

\subsection{Quotas} \label{appendix:sec-quotas}

\subsubsection{Semiconductors} \label{appendix:sec-semiconductors}

Hagstrum's model requires the density of states of the valence $D_v(\varepsilon)$ and conduction $D_c(\varepsilon)$ band as input, as well as the vacuum level. We calculate the density of states and vacuum level of \ce{Ge(111)} and \ce{Si(111)} using a DFT approach, as implemented in the Vienna Ab initio Simulation Package~\cite{Kresse1993, Kresse1994, Kresse1996, Kresse1996} (VASP). Within the Projector Augmented Wave~\cite{Blochl1994, Kresse1999} (PAW) formalism, the recommended number of valence electrons is included for both \ce{Ge} and \ce{Si}. The energy cutoff is set at 500~\si{\electronvolt} in order to obtain a well converged plane wave basis set, and the exchange correlation energy is calculated using the Perdew-Burke-Ernzerhof~\cite{Perdew1996} (PBE) functional. A well converged Monkhorst-Pack~\cite{Monkhorst1976} k-point mesh is used for sampling the Brillouin zone. We refer the reader to the supplementary information for more computational details.\\

To simulate a surface within the periodic boundary framework of VASP, it is conventional to take a slab approach, where a certain number of atomic layers are separated by a suitably large vacuum layer. For \ce{Si} and \ce{Ge}, it is well known that the (111) surfaces reconstruct, forming dimers at the surface with a 2x1 periodicity. We take the reconstructed structures from the supplementary material of De Waele et al.~\cite{DeWaele2016} and subsequently optimize the geometry using the computational parameters described in the previous paragraph. The slab consists of 14 atomic layers and at least 20 \si{\angstrom} of vacuum spacing is present. The vacuum level is obtained by averaging the one-electron electrostatic potential over planes parallel to the surface and determining the potential in the vacuum, which should be constant in case the vacuum layer is sufficiently thick. The work function $\phi$ of the surface is then calculated by comparing the vacuum level with the top of the valence band $\phi = \varepsilon_0 - \varepsilon_v$.

\resultsection{Parallelization}{https://mybinder.org/v2/gh/mbercx/jupyter/master?filepath=parallel\%2Fparallel_analysis.ipynb} 

\label{appendix:sec-parallel}\\
In order to speed up the calculations of the workflows, it is important to set reasonably good parallelization parameters for VASP (NPAR, KPAR). In order to do this, we have performed a whole series of tests, whose results are all available via the jupyter notebook corresponding to this section.

\printbibliography
\end{refsection}