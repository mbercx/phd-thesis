\begin{refsection} 

\chapter{Computational Details} \label{appendix:sec-computational} 
 
All calculations in this thesis were performed in the density functional 
theory (DFT) framework, as implemented in the Vienna Ab initio Simulation Package 
(\texttt{VASP})~\cite{Kresse1993, Kresse1994, Kresse1996}. This appendix starts with 
a brief description of this software package, as well as its input and 
output files (Sec.~\ref{appendix:sec-files}) and most important input 
parameters (Sec.~\ref{sec:input}). Next, the chapter details the 
settings used to obtain all of the results presented in this thesis, organised 
per chapter and section in the order they are presented. Finally, there are 
still two brief sections, one on parallelization tests 
(Sec.~\ref{appendix:sec-parallel}) and one on an issue with the way 
\texttt{VASP} calculates the real part of the dielectric tensor using 
the Kramers-Kronig relation.

\section{Vienna Ab initio Simulation Package} \label{appendix:sec-VASP} 
 
In order to solve the many-body problem using DFT, we need a software package 
that is able to implement the theory numerically on a computer cluster. 
Currently, there is a wide selection of such packages available to 
computational scientists, each with their respective advantages and 
disadvantages. 
 
The \texttt{VASP} package is particularly suited for materials science, 
using a plane wave basis set and offering PAW datasets for each atomic species. 
It can calculate an approximate solution to the many-body Schr\"odinger equation 
within the DFT formalism or the HF approximation, including the possibility of 
mixing to utilize hybrid functionals. VASP also employs a set of efficient 
iterative procedures to find the ground state of a system, and allows 
parallelization of the calculations on multi-core machines. 
 
This section presents a concise overview of the different files used by 
\texttt{VASP}, discussing their purpose, and take a closer look at the 
input parameters and their relation to the theory. 
 
\subsection{Files} \label{appendix:sec-files}
 
\texttt{VASP} uses four basic input files for its calculations, which 
must always be in the directory where \texttt{VASP} is executed:
 
\begin{itemize} 
 
\phantomsection \label{appendix:sec-INCAR} 
\item \href{https://www.vasp.at/wiki/index.php/INCAR}{\texttt{INCAR}}: 
Contains the input parameters for the calculation. Various settings can be 
adjusted according to the needs of the user through a large number of 
\textit{tags}, which are described in Section~\ref{appendix:sec-input}. This 
can be considered the most important input file, in the sense that it has 
the most diverse content and therefore has a lot of control over the 
simulation. Because of this, it is also more prone to be the cause of errors. 
 
\phantomsection \label{appendix:sec-POSCAR} 
\item \href{https://www.vasp.at/wiki/index.php/POSCAR}{\texttt{POSCAR}}: 
Contains the lattice vectors of the unit cell,as well as the atomic positions 
of the structure. The user is free to specify the atomic positions in cartesian 
or direct coordinates. In case selectivedynamics is used to fix certain atom 
coordinates in the unit cell, this is also indicated in this file. In order to 
make sure there are no mistakes in the POSCAR file, it is important to first 
visualize the structure.  
 
\phantomsection \label{appendix:sec-KPOINTS} 
\item \href{https://www.vasp.at/wiki/index.php/KPOINTS}{\texttt{KPOINTS}}: 
Defines the $\mathbf{k}$ point mesh (Sec.~\ref{dft:sec-kpoints}), either by 
explicitly entering all the points or using an automatically generated 
Monkhorst-Pack grid. For band structure calculations, there is the useful 
\textit{line mode}. Here the user canspecify certain (symmetry) lines along 
which to calculate the band structure of the crystal. These lines are described 
pairwise via the coordinates of their end points.
 
\phantomsection \label{appendix:sec-POTCAR} 
\item \href{https://www.vasp.at/wiki/index.php/POTCAR}{\texttt{POTCAR}}: 
Concatenation of the \link{dft:sec-PAW}{projector augmented wave} (PAW) 
datasets for the different elements present in the crystal structure. VASP 
supplies a set of POTCAR files for each element and supported functional, 
corresponding to different choices for the number of valence electrons and 
radii of the PAW sphere. It is important to make sure the order of the atoms 
is the same for the POTCAR and POSCAR file. 

\end{itemize}

Besides these four essential files, a few other files can serve as input files 
for the \texttt{VASP} calculation:

\begin{itemize}

\phantomsection \label{appendix:sec-STOPCAR} 
\item \href{https://www.vasp.at/wiki/index.php/STOPCAR}{\texttt{STOPCAR}}:
This file can be used to stop the calculation without killing the \texttt{VASP} 
process. By writing either \texttt{LABORT = True} or \texttt{LSTOP = True} to 
the \texttt{STOPCAR} file, the user can stop the calculation after the next 
electronic or ionic step, respectively.

\phantomsection \label{appendix:sec-CHGCAR} 
\item \href{https://www.vasp.at/wiki/index.php/CHGCAR}{\texttt{CHGCAR}}: 
The electronic charge density of the unit cell is written to this file. 
Although the file is an output file, it can be used as an input file to 
start a calculation with a desired charge density.

\phantomsection \label{appendix:sec-WAVECAR} 
\item \href{https://www.vasp.at/wiki/index.php/WAVECAR}{\texttt{WAVECAR}}: 
The plane wave coefficients of the wave functions are written to this file. 
Although the file is an output file, it can be used as an input file to 
start a calculation with a desired set of wave functions. Note that this 
is only possible if neither the number of bands nor the set of plane waves 
has changed.

\end{itemize} 

During the calculation, VASP produces a set of output files from which the 
user can extract the data necessary for his or her research. Here I present a  
(non-exhaustive) list of the most important VASP output files, besides the 
\vasp{CHGCAR} and \vasp{WAVECAR} already presented previously.
 
\begin{itemize} 
 
\phantomsection \label{appendix:sec-OUTCAR} 
\item \href{https://www.vasp.at/wiki/index.php/OUTCAR}{\texttt{OUTCAR}}: 
The general output file of VASP, a lot of information is printed in the this 
file during the calculation. The verbosity of the output is determined by the 
\href{https://www.vasp.at/wiki/index.php/NWRITE}{\texttt{NWRITE}} tag in the 
\vasp{INCAR} file. 

\phantomsection \label{appendix:sec-OSZICAR} 
\item \href{https://www.vasp.at/wiki/index.php/OSZICAR}{\texttt{OSZICAR}}: 
Presents an overview of the total energy at each SCF iteration, as well 
as some other properties interesting to the convergence of the calculation, 
for each ionic step in the case a geometry optimization is performed.

\phantomsection \label{appendix:sec-CONTCAR} 
\item \href{https://www.vasp.at/wiki/index.php/CONTCAR}{\texttt{CONTCAR}}: 
The final lattice vectors and atom positions of the unit cells are witten to 
this file. Obviously this is mostly important when performing a geometric 
optimization.

\phantomsection \label{appendix:sec-vasprun} 
\item \texttt{vasprun.xml}: 
XML formatted output file, written at the end of the calculation. A lot of 
output from other files is gathered into this one file, which makes it the 
most useful for post processing.

\phantomsection \label{appendix:sec-DOSCAR} 
\item \href{https://www.vasp.at/wiki/index.php/DOSCAR}{\texttt{DOSCAR}}:
Contains the density of states of the system, as well as the integrated DOS 
and the projected DOS, in case \href{https://www.vasp.at/wiki/index.php/LORBIT}{\texttt{LORBIT}} is set correctly.

\phantomsection \label{appendix:sec-IBZKPT} 
\item \href{https://www.vasp.at/wiki/index.php/IBZKPT}{\texttt{IBZKPT}}: 
The set of irreducible $\mathbf{k}$-points, along with their respective 
weights, can be found in this file.

\phantomsection \label{appendix:sec-EIGENVAL} 
\item \href{https://www.vasp.at/wiki/index.php/EIGENVAL}{\texttt{EIGENVAL}}: 
Details the Kohn-Sham eigenvalues for all $\mathbf{k}$-points, which can 
for example be used to plot the band structure. 

Note that some output is present in several files, but unfortunately 
\texttt{VASP} is not always entirely sensible in which output is printed where. 
An example here is the fact that although the \texttt{vasprun.xml} file 
contains \link{dft:sec-dielectric}{the dielectric tensor}, it does not 
contain \link{dft:sec-drude}{the plasma frequencies}. 

\end{itemize} 
 
\subsection{Input Parameters} \label{appendix:sec-input} 
 
This section presents some of the input tags which are set by 
the user to determine the specifics of the calculation. This is by no means 
an exhaustive list; we simply focus on a selection of tags that were 
especially relevant for the results presented in this thesis. For the complete 
list, we refer the reader to the 
\href{https://www.vasp.at/wiki/index.php/The_VASP_Manual}{VASP manual}. 
 
\begin{itemize} 
 
\phantomsection \label{appendix:sec-ENCUT} 
\item \href{https://cms.mpi.univie.ac.at/wiki/index.php/ENCUT}{\texttt{ENCUT}}: 
Energy cutoff (in \si{\electronvolt}) used for determining the size of the 
\link{dft:sec-planewave}{plane wave basis set}, as per 
Eq.~(\ref{dft:eq-energy_cutoff}).
 
\phantomsection \label{appendix:sec-PREC} 
\item \href{https://cms.mpi.univie.ac.at/wiki/index.php/PREC}{\texttt{PREC}}: 
Determines several settings that influence the precision of the calculations. 
First, the default value of the energy cutoff is increased for higher precision 
settings. Second, \texttt{PREC} sets the density of the grid used for the 
Fourier transformation. Finally, the precision of the representation of the 
\link{dft:sec-PAW}{PAW} projectors is set by PREC.
 
\phantomsection \label{appendix:sec-ALGO} 
\item \href{https://cms.mpi.univie.ac.at/wiki/index.php/ALGO}{\texttt{ALGO}}: 
Sets the algorithm used for the diagonalization of the Hamiltonian matrix 
when solving the \link{dft:sec-kohn_sham}{Kohn-Sham equations}.

\phantomsection \label{appendix:sec-EDIFF} 
\item \href{https://cms.mpi.univie.ac.at/wiki/index.php/EDIFF}{\texttt{EDIFF}}: 
Convergence criterion on the \link{dft:fig-SCF}{self-consistency cycle} 
during the electronic optimization, i.e. when determining the electron charge 
density.
 
\phantomsection \label{appendix:sec-ISMEAR} 
\item \href{https://cms.mpi.univie.ac.at/wiki/index.php/ISMEAR}{\texttt{ISMEAR}}: 
Specifies the \textit{smearing} method. Smearing is a technique that replaces the 
step function in the integral of the filled bands: 
\begin{equation}\label{appendix:eq-intsmear} 
\sum_n \frac{1}{\Omega_{BZ}}\int_{BZ} \epsilon_{n\mathbf{k}} 
\Theta(\epsilon_{n\mathbf{k}} - E_F) d\mathbf{k}, 
\end{equation} 
by a smooth function $f(\{\epsilon_{n\mathbf{k}}\})$. The main advantage of 
using smearing methods is that the integral in Eq.~(\ref{appendix:eq-intsmear}) can be 
calculated accurately using a relatively sparse $\mathbf{k}$-mesh 
(Sec.~\ref{dft:sec-kpoints}). The main smearing method used for the results 
presented in this thesis is \textit{Gaussian} smearing~\tocite (ISMEAR 
= 0). Another method used to solve the integral in Eq.~(\ref{appendix:eq-intsmear}) is 
the \textit{tetrahedron} method~\tocite (ISMEAR = -5), which 
linearly interpolates $\epsilon_{n\mathbf{k}}$ between two \textbf{k}-points. 
 
\phantomsection \label{appendix:sec-SIGMA} 
\item \href{https://cms.mpi.univie.ac.at/wiki/index.php/SIGMA}{\texttt{SIGMA}}: 
Smearing width used for the smearing of the occupies, as per the method 
specified with \texttt{ISMEAR}. Note that specifying \texttt{SIGMA} for 
the tetrahedron method makes little sense, as this method does not apply any 
smearing.

\phantomsection \label{appendix:sec-IBRION} 
\item \href{https://cms.mpi.univie.ac.at/wiki/index.php/ISIF}{\texttt{IBRION}}: 
Determines the algorithm for the geometry optimization. A common and stable 
choice here is the conjugate gradient algorithm~\tocite (\texttt{IBRION} = 2). 

\phantomsection \label{appendix:sec-ISIF} 
\item \href{https://cms.mpi.univie.ac.at/wiki/index.php/ISIF}{\texttt{ISIF}}: 
Specifies the degrees of freedom for the geometry optimization, e.g. whether to 
optimize only the atomic positions, or perform a full optimization of the 
structure.

\phantomsection \label{appendix:sec-EDIFFG} 
\item \href{https://cms.mpi.univie.ac.at/wiki/index.php/EDIFFG}{\texttt{EDIFFG}}: 
Sets the value for the convergence condition of the geometry optimization. The 
condition can be either applied to the total energy by setting a positive value, 
or the forces, when a negative value is provided.
 
\phantomsection \label{appendix:sec-LOPTICS} 
\item \href{https://cms.mpi.univie.ac.at/wiki/index.php/LOPTICS}{\texttt{LOPTICS}}: 
Boolean setting that indicates that the 
\link{dft:sec-dielectric}{frequency dependent dielectric tensor} should be 
calculated.

\phantomsection \label{appendix:sec-CSHIFT} 
\item \href{https://cms.mpi.univie.ac.at/wiki/index.php/CSHIFT}{\texttt{CSHIFT}}: 
Complex shift used in the \link{dft:eq-kramers}{Kramers-Kronig relation} used 
to calculate the real part of the dielectric tensor.

\phantomsection \label{appendix:sec-NBANDS} 
\item \href{https://cms.mpi.univie.ac.at/wiki/index.php/NBANDS}{\texttt{NBANDS}}: 
Number of bands to include in the calculation. \texttt{VASP} includes a limited 
number of empty bands by default, but in order to calculate the optical properties 
or density of states over a larger energy range, this number should be increased 
to at least 2-3 times the default value.
s
\phantomsection \label{appendix:sec-NEDOS} 
\item \href{https://cms.mpi.univie.ac.at/wiki/index.php/NEDOS}{\texttt{NEDOS}}: 
Number of points in the energy mesh for the calculation of the density of states 
and dielectric tensor.
 
\phantomsection \label{appendix:sec-ISPIN} 
\item \href{https://cms.mpi.univie.ac.at/wiki/index.php/ISPIN}{\texttt{ISPIN}}: 
Specifies the spin-polarization setting, i.e. 1 or 2. The default is to perform 
a non-spin-polarized calculation (\texttt{ISPIN} = 1). Using \texttt{ISPIN} = 2 
starts a spin-polarized calculation, i.e. for a system with collinear spins.
 
\phantomsection \label{appendix:sec-MAGMOM} 
\item \href{https://cms.mpi.univie.ac.at/wiki/index.php/MAGMOM}{\texttt{MAGMOM}}: 
Allows the user to set the magnetic moments of all the atoms in the unit cell, 
when performing a spin-polarized calculation with collinear spins (\texttt{ISPIN} = 1).
For non-collinear calculations, the magnetization density is a vectorial quantity, 
and the components of the magnetiziation density should be provided for each atom, 
with respect to the spin quantization axis (see 
\href{https://cms.mpi.univie.ac.at/wiki/index.php/SAXIS}{SAXIS}).

\phantomsection \label{appendix:sec-LHFCALC} 
\item \href{https://cms.mpi.univie.ac.at/wiki/index.php/LHFCALC}{\texttt{LHFCALC}}: 
Boolean tag that indicates that the \link{dft:sec-hybrid}{Hartree-Fock/DFT hybrid 
calculations} should be performed. 

\phantomsection \label{appendix:sec-AEXX} 
\item \href{https://cms.mpi.univie.ac.at/wiki/index.php/AEXX}{\texttt{AEXX}}: 
Determines the mixing parameter for the \link{dft:sec-hybrid}{hybrid calculations}, 
i.e. the fraction $a$ of Hartree-Fock exact exchange energy that should be 
included for the short range interaction.

\phantomsection \label{appendix:sec-HFSCREEN} 
\item \href{https://cms.mpi.univie.ac.at/wiki/index.php/HFSCREEN}{\texttt{HFSCREEN}}: 
Specifies the separation parameter $\omega$ for 
\link{dft:sec-hybrid}{hybrid calculations}, i.e. beyond what distance the 
interaction energy is considered to be long range instead of short range.

\end{itemize} 

\section{Results} \label{appendix:sec-results} 

The projector augmented wave (PAW) method [15] was used to make a 
distinction between the core and valence electrons, with the standard VASP 
recommended choice for the number of valence electrons.  

\subsection{Solar Cells} \label{appendix:sec-solar} 
 
We make a selection of ten compounds for which we can compare the calculated 
efficiency of the CA phase with the CH results of Yu and Zunger~\cite{Yu2012}. 
The CA and CH structure are studied using a first-principles approach within 
the Density Functional Theory (DFT) formalism, as implemented in the Vienna Ab 
initio Simulation Package~\cite{Kresse1993, Kresse1996, Kresse1996b} (VASP). 
The Projector Augmented Wave (PAW) method~\cite{Blochl1994} is applied, and the 
electrons that are treated as valence electrons are underlined in 
Table~\ref{tab:valElec}. The exchange-correlation functional is calculated 
using the Generalized Gradient Approximation (GGA) of Perdew-Burke-Ernzerhof 
(PBE)~\cite{Perdew1996}. The energy cutoff for the plane wave basis is set to 
350 eV, and a 4$\times$4$\times$4 Monkhorst-Pack~\cite{Monkhorst1976} (MP) 
mesh is used for sampling the first Brillouin zone. Electronic convergence is 
obtained when the energy difference between two electronic steps is smaller 
than $10^{-4}$~\si{\electronvolt}. The structure is considered converged when 
the forces on the atoms are all below 
$10^{-2}$~\si{\electronvolt}/\si{\angstrom}. \\\vspace{0.5em} 
Because an accurate band gap is important for the correct evaluation of the 
efficiency, we perform single shot G$_0$W$_0$~\cite{Hybertsen1985} 
calculations on top of HSE06~\cite{Heyd2006}. However, in order to accurately 
update the quasiparticle energies within the G$_0$W$_0$ approximation, it is 
necessary to consider the semi-core electrons as valence electrons within the 
PAW framework~\cite{Fuchs2007}. Hence, we treat the 3$s$, 3$p$ and 3$d$ (4$s$, 
4$p$ and 4$d$) orbitals as valence states for the Ga (In) atoms for the 
G$_0$W$_0$@HSE06 calculations of the band gap. In addition, we use a well 
converged 8$\times$8$\times$8 MP mesh, an increased energy cutoff of 
400~\si{\electronvolt} and a large amount of unoccupied bands (600 in total). 
\\\vspace{0.5em} 
The optical properties are calculated within the Random Phase Approximation 
(RPA), using the long wavelength expression for the imaginary part of the 
dielectric tensor~\cite{Gajdos2006, Harl2007}. The real part of the dielectric 
tensor is determined using the Kramers-Kronig relation\footnote{The 
Kramers-Kronig relation is calculated by VASP using a complex shift 
(``CSHIFT''). After calculating the real part, however, VASP also recalculates 
the corresponding imaginary part. Since the complex shift introduces a 
broadening, this causes an earlier onset of the imaginary part, and 
consequently in the absorption coefficient. In order to prevent this, we 
commented out the line in the VASP code that recalculates the imaginary 
part.}. In order to get an accurate description of the energy levels, the 
exchange-correlation energy is calculated with the HSE06 functional, which has 
been reported~\cite{Wan2013} to produce optical properties close to those 
obtained from experiment for CuIn(S$_x$Se$_{x-1}$)$_2$. We found that it is 
enough to sample the Brillouin zone using a 12$\times$12$\times$12 MP mesh to 
obtain a converged dielectric tensor. The number of unoccupied bands is 
increased to at least three times the number of occupied bands. Because of the 
tetragonal symmetry of the CA structure, the resulting dielectric tensor is 
diagonal and has two independent components $\varepsilon_{xx}$ 
(=$\varepsilon_{yy}$) and $\varepsilon_{zz}$. Since we make no assumptions 
about the direction from which the photons enter the absorber layer, we 
average the diagonal components to derive the dielectric function $\varepsilon 
(E) = \varepsilon^{(1)} (E) + i \varepsilon^{(2)} (E) $ at energy $E$. 
Finally, in order to obtain a more accurate onset of the absorption spectrum, 
we shift the imaginary part of the dielectric function to the G$_0$W$_0$@HSE06 
band gap, and recalculate the real part using the Kramers-Kronig relations.  
 
\begin{table}[htbp] 
\centering 
\setlength{\captionmargin}{20pt} 
\renewcommand{\arraystretch}{1.2} 
\caption{\label{tab:valElec}Electron configuration of the atoms.} 
\begin{tabular}{c@{\hskip 2 em}l}\hline 
Element & Configuration \\\hline 
Cu & [Ar] \underline{3d$^{10}$4s$^1$} \\ 
Ag &[Kr] \underline{4d$^{10}$5s$^1$} \\ 
Ga &[Ar] 3d$^{10}$\underline{4s$^2$4p$^1$}\\ 
In &[Kr] 4d$^{10}$\underline{5s$^{2}$4p$^1$} \\ 
S &[Ne] \underline{3s$^2$3p$^4$} \\ 
Se &[Ar]  3d$^{10}$\underline{4s$^{2}$4p$^4$} \\ 
Te &[Kr] 4d$^{10}$\underline{5s$^2$5p$^4$}\\ 
\hline 
\end{tabular} 
\end{table} 
 
\subsection{Li-ion Batteries} 
 
The wave functions of the valence electrons are expanded in a plane wave basis 
set, using a high energy cutoff equal to 500 eV, which is advisable for 
structures containing oxygen. The bulk structures were optimized with the 
hybrid HSE06 functional [16]. A suitably converged Monkhorst-Pack mesh [17] 
was chosen for the k-point sampling of the Brillouin zone, with a k-mesh 
spacing smaller than 0.05 \AA. Geometry optimizations were performed with a 
Gaussian smearing of 0.05 eV for semiconductor and insulators, and a 
Methfessel-Paxton [18] smearing of 0.2 eV for metallic structures, followed by 
a self consistent field calculation using the tetrahedron method [19], for an 
accurate calculation of the total energies. The convergence criterion on the 
electronic optimization is set at 10" eV, and 10"' eV for the geometric 
optimization. As Ir is known to exhibit a strong spin-orbit interaction, 
non-collinear calculations including spin-orbit coupling were performed for 
calculating the magnetic moment and density of states of LixIrO3. The dimer 
reaction energy and kinetics of both compounds were calculated in a 2x2x2 
supercell, where we switched to the PBE+U functional [20,21] in order to make 
the dimer screening computationally feasible. A range of choices for the U 
parameter were tested to closely match the magnetic moments and lattice 
constants of the HSE06 calculation for the bulk structures, and we settled on 
3.9 eV for Mn and 4.0 eV for Ir. Activation energies were calculated using the 
nudged elastic band (NEB) method [22].  
 
\begin{table}[h] 
\centering 
\caption{Overzicht invloed U-waarden voor O1-\ce{Li_{0.5}MnO_3}} 
  \begin{tabular}{rrrrrr} 
    \toprule 
    U (eV) & Mn-3d ($\mu_B$) & O-2p ($\mu_B$) & $|\vec{a}|$ (\si{\angstrom}) & 
$|\vec{c}|$ (\si{\angstrom}) & $\Delta E$ dimer (meV) \\ 
    \midrule 
    1.0 & 2.319 & -0.248 & 9.779 & 8.483 &  \\  
    2.0 & 2.660 & -0.348 & 9.886 & 8.250 & +153 \\ 
    3.0 & 2.874 & -0.408 & 9.955 & 8.214 & +27 \\  
    3.5 & 2.977 & -0.438 & 9.978 & 8.223 & +6 \\  
    4.0 & 3.068 & -0.463 & 10.000 & 8.231 & -70 \\  
    4.5 & 3.159 & -0.489 & 10.033 & 8.217 & -138 \\  
    5.0 & 3.252 & -0.515 & 10.068 & 8.216 & -181 \\   
    6.0 & 3.444 & -0.570 & 10.147 & 8.204 & -343 \\  
   \bottomrule 
   \end{tabular} 
\end{table} 
 
\subsubsection{Structure and Li-configurations} \label{appendix:sec-structure} 
All calculations are performed using the \textbf{PBE+U functional  
functional}, where an effective U-correction of 3.9 eV is applied to the $3d$ orbitals of 
Mn. The wave functions of the valence electrons are expanded in a plane wave 
basis set, using a high \textbf{energy cutoff equal to 500 eV}, which is 
advisable for structures containing oxygen. A suitably converged 
Monkhorst-Pack mesh was chosen for the k-point sampling of the Brillouin zone, 
with a k-mesh spacing smaller than 0.05 \AA. Geometry optimizations were 
performed with a \textbf{Gaussian smearing of 0.05 eV} for semiconductor and 
insulators, and a Methfessel-Paxton smearing of 0.2 eV for metallic 
structures, followed by a self consistent field calculation using the 
tetrahedron method, for an accurate calculation of the total energies. The 
convergence criterion on the electronic optimization is set at $10^{-6}$~eV, 
and $10^{-3}$ eV for the geometric optimization. 
 
\subsubsection{Oxidation} \label{appendix:sec-oxidation} 
All calculations were performed with the \textbf{hybrid HSE06 functional} 
[16].The wave functions of the valence electrons are expanded in a plane wave 
basis set, using a high \textbf{energy cutoff equal to 500 eV}, which is 
advisable for structures containing oxygen. A suitably converged 
Monkhorst-Pack mesh [17] was chosen for the k-point sampling of the Brillouin 
zone, with a k-mesh spacing smaller than 0.05 \AA. Geometry optimizations were 
performed with a \textbf{Gaussian smearing of 0.05 eV} for semiconductor and 
insulators, and a Methfessel-Paxton [18] smearing of 0.2 eV for metallic 
structures, followed by a self consistent field calculation using the 
tetrahedron method [19], for an accurate calculation of the total energies. 
The convergence criterion on the electronic optimization is set at 
$10^{-6}$~eV, and $10^{-3}$ eV for the geometric optimization. As Ir is known 
to exhibit a strong spin-orbit interaction, non-collinear calculations 
including spin-orbit coupling were performed for calculating the magnetic 
moment and density of states of LixIrO3. The dimer reaction energy and 
kinetics of both compounds were calculated in a 2x2x2 supercell, where we 
switched to the PBE+U functional [20,21] in order to make the dimer screening 
computationally feasible. A range of choices for the U parameter were tested 
to closely match the magnetic moments and lattice constants of the HSE06 
calculation for the bulk structures, and we settled on 3.9 eV for Mn and 4.0 
eV for Ir. Activation energies were calculated using the nudged elastic band 
(NEB) method [22].  
 
\subsubsection{Dimer} \label{appendix:sec-dimer} 
All calculations were performed with the PBE+U functional [20,21]. The wave 
functions of the valence electrons are expanded in a plane wave basis set, 
using a high \textbf{energy cutoff equal to 500 eV}, which is advisable for 
structures containing oxygen. A suitably converged Monkhorst-Pack mesh [17] 
was chosen for the k-point sampling of the Brillouin zone, with a k-mesh 
spacing smaller than 0.05 \AA. Geometry optimizations were performed with a 
\textbf{Gaussian smearing of 0.05 eV} for semiconductor and insulators, and a 
Methfessel-Paxton [18] smearing of 0.2 eV for metallic structures, followed by 
a self consistent field calculation using the tetrahedron method [19], for an 
accurate calculation of the total energies. The convergence criterion on the 
electronic optimization is set at $10^{-6}$~eV, and $10^{-3}$ eV for the 
geometric optimization. As Ir is known to exhibit a strong spin-orbit 
interaction, non-collinear calculations including spin-orbit coupling were 
performed for calculating the magnetic moment and density of states of 
LixIrO3. The dimer reaction energy and kinetics of both compounds were 
calculated in a 2x2x2 supercell, where we switched to  A range of choices for 
the U parameter were tested to closely match the magnetic moments and lattice 
constants of the HSE06 calculation for the bulk structures, and we settled on 
3.9 eV for Mn and 4.0 eV for Ir. Activation energies were calculated using the 
nudged elastic band (NEB) method [22].  
 
\subsection{Ion-Induced Secondary Electron Emission} \label{appendix:sec-quotas} 
 
\subsubsection{Semiconductors} \label{appendix:sec-semiconductors} 
 
Hagstrum's model requires the density of states of the valence 
$D_v(\varepsilon)$ and conduction $D_c(\varepsilon)$ band as input, as well as 
the vacuum level. We calculate the density of states and vacuum level of 
\ce{Ge(111)} and \ce{Si(111)} using a DFT approach, as implemented in the 
Vienna Ab initio Simulation Package~\cite{Kresse1993, Kresse1994, Kresse1996, 
Kresse1996} (VASP). Within the Projector Augmented Wave~\cite{Blochl1994, 
Kresse1999} (PAW) formalism, the recommended number of valence electrons is 
included for both \ce{Ge} and \ce{Si}. The energy cutoff is set at 
500~\si{\electronvolt} in order to obtain a well converged plane wave basis 
set, and the exchange correlation energy is calculated using the 
Perdew-Burke-Ernzerhof~\cite{Perdew1996} (PBE) functional. A well converged 
Monkhorst-Pack~\cite{Monkhorst1976} k-point mesh is used for sampling the 
Brillouin zone.
 
To simulate a surface within the periodic boundary framework of VASP, it is 
conventional to take a slab approach, where a certain number of atomic layers 
are separated by a suitably large vacuum layer (See~\ref{automation:sec-surface}). 
For \ce{Si} and \ce{Ge}, it is 
well known that the (111) surfaces reconstruct, forming dimers at the surface 
with a 2x1 periodicity. We take the reconstructed structures from the 
supplementary material of De Waele et al.~\cite{DeWaele2016} and subsequently 
optimize the geometry using the computational parameters described in the 
previous paragraph. The slab consists of 14 atomic layers and at least 20 
\si{\angstrom} of vacuum spacing is present. The vacuum level is obtained by 
averaging the one-electron electrostatic potential over planes parallel to the 
surface and determining the potential in the vacuum, which should be constant 
in case the vacuum layer is sufficiently thick. The work function $\phi$ of 
the surface is then calculated by comparing the vacuum level with the top of 
the valence band $\phi = \varepsilon_0 - \varepsilon_v$. 

\subsubsection{Metals} \label{appendix:sec-metals} 

Hagstrum's model requires the density of states of the occupied and unoccupied  
($D_v(\varepsilon)$ and $D_c(\varepsilon)$) states as input, as well as 
the vacuum level. We calculate the density of states and vacuum level of 
all metal surfaces using a DFT approach, as implemented in the 
Vienna Ab initio Simulation Package~\cite{Kresse1993, Kresse1994, Kresse1996, 
Kresse1996} (VASP). Within the Projector Augmented Wave~\cite{Blochl1994, 
Kresse1999} (PAW) formalism, the recommended number of valence electrons is 
included for all metals. The energy cutoff is set at 
500~\si{\electronvolt} in order to obtain a well converged plane wave basis 
set, and the exchange correlation energy is calculated using the 
Perdew-Burke-Ernzerhof~\cite{Perdew1996} (PBE) functional. A well converged 
Monkhorst-Pack~\cite{Monkhorst1976} k-point mesh is used for sampling the 
Brillouin zone.

To simulate a surface within the periodic boundary framework of VASP, it is 
conventional to take a slab approach, where a certain number of atomic layers 
are separated by a suitably large vacuum layer (See~\ref{automation:sec-surface}).
We take the structures of all surfaces from the 
supplementary material of De Waele et al.~\cite{DeWaele2016} and subsequently 
optimize the geometry using the computational parameters described in the 
previous paragraph. The slab consists of 14 atomic layers and at least 20 
\si{\angstrom} of vacuum spacing is present. The vacuum level is obtained by 
averaging the one-electron electrostatic potential over planes parallel to the 
surface and determining the potential in the vacuum, which should be constant 
in case the vacuum layer is sufficiently thick. The work function $\phi$ of 
the surface is then calculated by comparing the vacuum level with the top of 
the valence band $\phi = \varepsilon_0 - \varepsilon_v$. 

The optical properties of the bulk are calculated within the Random Phase 
Approximation (RPA), using the long wavelength expression for the imaginary 
part of the dielectric tensor~\cite{Gajdos2006, Harl2007} 
(see Sec.~\ref{dft:sec-dielectric}). The real part of the dielectric tensor is 
determined using the Kramers-Kronig  relation. For the damping parameter in 
the Drude expression of the \link{dft:sec-drude}{intraband part of the 
dielectric tensor} a value of 50~\si{\milli\electronvolt} is used.

\resultsection{Parallelization \label{appendix:sec-parallel}}{https://mybinder.org/v2/gh/mbercx/jupyter/master?filepath=parallel\%2Fparallel_analysis.ipynb} 

In order to speed up the calculations of the workflows, it is important to set 
reasonably good parallelization parameters for VASP (NPAR, KPAR). In order to 
do this, we have performed a whole series of tests, whose results are all 
available via the jupyter notebook corresponding to this section. 
 
\resultsection{\texttt{CSHIFT} \label{appendix:sec-cshift}}{} 

Kramers-Kronig and stuff 

\chapter{Plasmonic Excitations} \label{appendix:sec-plasmons} 

\section{Poisson Processes} \label{sec:appendix-poisson} 

A Poisson process is defined as (taken from~\cite{MITopencourseware}):

\begin{adjustwidth}{1em}{0em} 
\textit{A Poisson process is a renewal process in which the interarrival 
intervals $T_p$ follow an exponential distribution function; i.e., for some 
real $\lambda_p > 0$, $T_p$ has the probability density function}
\begin{equation}
f_{T_p} (t) = 
\begin{cases} 
\lambda_p e^{-\lambda_p t} &\text{  for } t \geq 0 \\ 
0 &\text{  for } t < 0
\end{cases}
\end{equation}
\end{adjustwidth} 
Let's start with some basic expressions that facilitate the derivations of the 
plasmon probabilities. First, note that the probability density function is 
properly normalized to 1:
\begin{eqnarray*}
\int_{-\infty}^\infty \lambda_p e^{-\lambda_p t} dt &=& \int_0^\infty \lambda_p e^{-\lambda_p t} dt \\
                     &=& \lambda_p \left[ \frac{1}{-\lambda_p} e^{-\lambda_p t} \right]_0^\infty \\
                     &=& -(0 - 1) = 1.
\end{eqnarray*}
The probability of the interval of process $p$ being smaller than some 
specified interval $T$ is:
\begin{eqnarray}
\text{Pr}\{T_p \leq T\} &=& \int_0^T \lambda_p e^{-\lambda_p t} dt \nonumber \\
                     &=& \lambda_p \left[ \frac{1}{-\lambda_p} e^{-\lambda_p t} \right]_0^T \nonumber \\
                     &=& -(e^{-\lambda_p T} - 1) = 1 - e^{-\lambda_p T},
\end{eqnarray}
which is simply the cumulative distribution function of the Poisson process. 
Similarly, the probability of the interval being larger than some 
specified interval $T$ is:
\begin{eqnarray}
\text{Pr}\{T_p > T\} &=& 1 - \text{Pr}\{T_p \leq T\} \nonumber \\
                     &=& 1 - (1 - e^{-\lambda_p T}) = e^{-\lambda_p T}
\end{eqnarray}
For two competing Poisson processes with rates $\lambda_1$ and 
$\lambda_2$, the probability of the interval $T_1$ being smaller than $T_2$ is:
\begin{eqnarray} 
\text{Pr}\{T_1 < T_2\} &=& \int_0^\infty \text{Pr}\{T_1 < T_2 | T_1 = t \} \lambda_1 e^{-\lambda_1 t} dt \nonumber \\
                       &=& \int_0^\infty \text{Pr}\{T_2 > t\} \lambda_1 e^{-\lambda_1 t} dt \nonumber \\
                       &=& \int_0^\infty e^{-\lambda_2 t} \lambda_1 e^{-\lambda_1 t} dt \nonumber \\
                       &=& \lambda_1 \int_0^\infty e^{- (\lambda_1 + \lambda_2) t} dt \nonumber \\
                       &=& \frac{\lambda_1}{-(\lambda_1 + \lambda_2)} (0 - 1) \nonumber \\
                       &=& \frac{\lambda_1}{\lambda_1 + \lambda_2} \label{appendix:eq-competing}
\end{eqnarray} 

Equation~(\ref{quotas:eq-sp_poisson}) is now fairly easy to derive. Consider two 
Poisson processes, one for the surface plasmon excitation ($sp$) and one for
the Auger neutralization ($aug$):
\begin{eqnarray*} 
f_{T_{sp}}(E_{sp}, t) &=& \lambda_{sp} (E_{sp}) e^{-\lambda_{sp}(E_{sp}) t}\\ 
f_{T_{aug}}(t) &=& \lambda_{aug} e^{-\lambda_{aug} t}.
\end{eqnarray*}
The probability of a surface plasmon excitation is equal to that of the 
interval $T_{sp}$ being smaller than $T_{aug}$:
\begin{eqnarray} 
P_{sp}(E_{sp}) &=& \text{Pr}\{T_{sp} < T_{aug}\} \nonumber \\
               &\stackrel{(\ref{appendix:eq-competing})}{=}& \frac{\lambda_{sp} (E_{sp}) }{\lambda_{sp} (E_{sp})  + \lambda_{aug}} \\
\end{eqnarray} 

For the volume plasmons, we have to consider multiple competing Poisson processes. 
First, note that two independent Poisson processes with rates $\lambda_1$ and 
$\lambda_2$ can be merged into a new Poisson process with rate $\lambda' = \lambda_1 
+ \lambda_2$. So, if we consider a third process with rate $\lambda_3$, the probability 
of the interval of this process being smaller than that of process $1$ and $2$ is:
\begin{eqnarray} 
\text{Pr}\{T_3 < T'\} &=& \frac{\lambda_3 }{\lambda' + \lambda_3} \nonumber \\
                      &=& \frac{\lambda_3 }{\lambda_1  + \lambda_2 + \lambda_3}.
\end{eqnarray} 
By extension, the probability of a Poisson process with rate $\lambda_n$ having 
the shortest interval among a set of independent processes $\{\lambda_i\}$, with 
$i = 1, ..., N$, becomes:
\begin{equation}
\text{Pr}\{T_n < \text{min}(T_1, T_2, ..., T_{n-1}, T_{n+1}, ..., T_N)\} = \frac{\lambda_n}{\lambda_1  + ... + \lambda_N}.
\end{equation}
Within our model, an electron with energy $\varepsilon$ can excite plasmons up to 
energy $\varepsilon - \varepsilon_F$, and the rate of exciting a volume plasmon with 
energy $E_{vp}$ is proportional to the energy loss function $L(E)$:
\begin{equation} 
\lambda_{vp}(E_{vp}) = 
\begin{cases} 
c_{vp} \cdot L(E_{vp}) &\text{  for } 0 < E_{vp} \leq \varepsilon - \varepsilon_F \\
0 &\text{  for } E_{vp} \leq 0 \text{ or } E_{vp} > \varepsilon - \varepsilon_F.
\end{cases}
\end{equation} 
In this case, we have a continuous set of independent Poisson processes, each with 
rate equal to $\lambda_{vp}(E_{vp})$, so the probability distribution of first
exciting a volume plasmon with energy $E_{vp}$ becomes:
\begin{equation}
P_{vp}(\varepsilon, E_{vp}) = \text{Pr}\{T_{vp}(E_{vp}) = \text{min}(T_{vp}(E)\} = \frac{\lambda_{vp}(E_{vp})}{\int_0^{\varepsilon - \varepsilon_F} \lambda_{vp}(E) dE}.
\end{equation}
Note that this distribution is properly normalized to 1:
\begin{eqnarray*}
\int_0^\infty P_{vp}(\varepsilon, E_{vp}) dE_{vp} &=& \int_0^\infty P_{vp}(\varepsilon, E_{vp}) dE_{vp}  \\
&=& \frac{\int_0^\infty \lambda_{vp}(E_{vp}) dE_{vp}}{\int_0^{\varepsilon - \varepsilon_F} \lambda_{vp}(E) dE} \\
&=& \frac{c_{vp} \int_0^{\varepsilon - \varepsilon_F} L(E_{vp}) dE_{vp}}{c_{vp} \int_0^{\varepsilon - \varepsilon_F} L(E) dE} = 1 \\
\end{eqnarray*}
However, we also have to consider the average travel interval of the electrons 
$t_e$. As the total rate of the volume plasmon excitation process is 
$\int_0^{\varepsilon - \varepsilon_F} \lambda_{vp}(E) dE$, the probability of 
any volume plasmon being excited in the interval $t_e$ is:
\begin{equation}
\text{Pr}\{\text{min}(T_{vp}(E)) < t_e \} = 1 - e^{-(\int_0^{\varepsilon - \varepsilon_F} \lambda_{vp}(E) dE) t_e}.
\end{equation}
Combining both leads to the probability distribution of a volume plasmon 
excitation for an electron with average travel interval $t_e$:
\begin{eqnarray}
P_{vp}(\varepsilon, E_{vp}) dE_{vp} &=& \frac{\lambda_{vp}(E_{vp})dE_{vp}}{\int_0^{\varepsilon - \varepsilon_F} \lambda_{vp}(E) dE }\left[1-e^{-\left(\int_0^{\varepsilon - \varepsilon_F} \lambda_{vp}(E_{vp}) dE\right) t_e }\right] \nonumber \\
&=& \frac{c_{vp} L(E_{vp})dE_{vp}}{\int_0^{\varepsilon - \varepsilon_F} c_{vp} L(E) dE }\left[1-e^{-\left(\int_0^{\varepsilon - \varepsilon_F} c_{vp} L(E) dE\right) t_e }\right] \nonumber \\
&=& \frac{L(E_{vp})dE_{vp}}{\int_0^{\varepsilon - \varepsilon_F} L(E) dE 
}\left[1-e^{-\left(\int_0^{\varepsilon - \varepsilon_F} L(E) dE\right) c_{vp} t_e 
}\right] \label{appendix:eq-vp_prob}
\end{eqnarray}
 
\section{Volume Plasmons} 
 
Consider the distribution of excited electrons $N_i(\varepsilon)$. The 
probability that an excited electron at energy $\varepsilon$ induces a volume 
plasmon of energy $E_{vp} < \varepsilon - \varepsilon_F$ is given by Eq.~(\ref{appendix:eq-vp_prob}). The fraction of electrons which excite a volume plasmon is then equal to: 
\begin{equation} 
    N_{vp}^{-}(\varepsilon) = \int_0^\infty N_i(\varepsilon) 
P_{vp}(\varepsilon, E_{vp}) dE_{vp}. 
\end{equation} 
These are substracted from the excited electron density. However, these 
electrons are not lost, they have simply lost an energy equal to the volume 
plasmon energy $E_{vp}$. Hence, we have to add the following density to the 
excited density again: 
\begin{equation} 
    N_{vp}^{+}(\varepsilon) = \int_0^\infty N_i(\varepsilon + E_{vp}) 
P_{vp}(\varepsilon, E_{vp}) dE_{vp}. 
\end{equation} 
The story doesn't end there, however. Volume plasmons do not couple with 
transversal waves~\cite{Maier2007}, and hence cannot decay radiatively. Most likely, the 
plasmon will decay as a single electron excitation. In order to model this, we 
first calculate the distribution of plasmons with an energy $E_{vp}$: 
\begin{equation} 
    D_{vp} (E_{vp})= \int_0^\infty N_i(\varepsilon) P_{vp}(\varepsilon, 
E_{vp}) d\varepsilon 
\end{equation} 
When these plasmons decay, they release an energy equal to $E_vp$. In other 
words, this distribution should be convoluted with the density of the occupied 
states $D_v(\varepsilon)$ in order to obtain the density of excited electrons 
due to volume plasmon decay: 
\begin{eqnarray} 
    N_{vp}^{d}(\varepsilon) &=& \frac{1}{n_{vp}}(D_v * D_{vp}) (\varepsilon) 
\\ 
    &=& \frac{1}{n_{vp}} \int_{\Delta\varepsilon_v} D_v(\varepsilon_1) 
D_{vp}(\varepsilon - \varepsilon_1) d\varepsilon_1 \\ 
    &=& \frac{1}{n_{vp}} \int_{\Delta\varepsilon_v} D_v(\varepsilon_1)  
    \int_0^\infty N_i(\varepsilon_2) P_{vp}(\varepsilon_2, \varepsilon - 
\varepsilon_1) d\varepsilon_2 d\varepsilon_1  
\end{eqnarray} 
where $n_{vp}$ is a normalization factor. Of course, this distribution should 
be normalized to the number of excited plasmons, i.e.: 
\begin{equation} 
    n_{vp} = \int_0^\infty N_{vp}^{-}(\varepsilon) d\varepsilon 
\end{equation} 

\section{Surface Plasmons} 
 
Hagstrum's Auger transform can be written as a convolution: 
\begin{equation} 
\begin{aligned} 
T\left[\frac{\varepsilon + \varepsilon_0 - E_i}{2}\right] &= 
\int_{\Delta\varepsilon_v}\int_{\Delta\varepsilon_v} D_v (\varepsilon_1) D_v 
(\varepsilon_2) \delta(\varepsilon - \varepsilon_1 - \varepsilon_2 + 
\varepsilon_0 - E_i) d\varepsilon_1 d\varepsilon_2. \\ 
&= \int_{\Delta\varepsilon_v} D_v (\varepsilon_1) D_v (\varepsilon + 
\varepsilon_0 - E_i - \varepsilon_1) d\varepsilon_1 \\  
&= (D_v * D_v) (\varepsilon + \varepsilon_0 - E_i) 
\end{aligned} 
\end{equation} 
which can also be written as: 
\begin{equation} 
T\left[\frac{\varepsilon + \varepsilon_0 - E_i}{2}\right] = D_v(\varepsilon) * 
D_v(\varepsilon + \varepsilon_0 - E_i) 
\end{equation} 
 
Here, $D_E(\varepsilon) = D_v(\varepsilon + \varepsilon_0 - E_i)$ 
corresponds to the distribution of energies released by the neutralization of 
the incoming ion\footnote{This should be more clear when we substitute e.g. 
$\varepsilon = 0$ in $D_E(\varepsilon)$. We then obtain $D_v(\varepsilon_0 - 
E_i)$, i.e. the number of electrons in the valence band that are at an equal 
depth (in relation to the vacuum level) as the lowest unoccupied state of the 
incoming ion. Clearly, electrons at this level do not pass energy through any 
Auger process, as this is a resonant neutralization. Another way of 
understanding this is that an electron at energy level $\varepsilon_0 - E_i + 
\varepsilon$ releases an energy of $\varepsilon$ when it neutralizes the 
incoming ion.}.  In order to include surface plasmon excitations, we consider 
the probability that a plasmon with energy $E_{sp}$ is excited instead of an 
Auger process: 
\begin{equation}
P_{sp}(E_{sp}) = \frac{ \frac{c_{sp}}{\lambda_{aug}} 
\text{Im}\left[-\frac{1}{\epsilon(E_{sp}) + 1}\right]}{\frac{c_{sp}}{\lambda_{aug}} 
\text{Im}\left[-\frac{1}{\epsilon(E_{sp}) + 1}\right] + 1} 
\end{equation} 
As the surface plasmon excitation process is considered to be resonant, 
$E_{sp}$ must correspond to the energy released in the neutralization of the 
incoming ion, i.e. $\varepsilon$ in $D_E(\varepsilon)$. The leftover 
distribution of energies passed to other valence electrons through an Auger 
process is: 
\begin{equation} 
    D_{aug}(\varepsilon) = D_E(\varepsilon) (1 - P_{sp}(\varepsilon)) 
\end{equation} 
We substitute this distribution in the expression for the Auger transform: 
\begin{equation} 
\begin{aligned} 
T'\left[\frac{\varepsilon + \varepsilon_0 - E_i}{2}\right] &= D_v(\varepsilon) 
* D_{aug}(\varepsilon) \\ &= D_v(\varepsilon) * (D_v(\varepsilon + 
\varepsilon_0 - E_i) (1 - P_{sp}(\varepsilon))) 
\end{aligned} 
\end{equation} 
And use this Auger transform instead of the original one in the nominator of 
Eq.~(\ref{quotas:eq-excite}). Note that the normalization in the denominator 
of Eq.~(\ref{quotas:eq-excite}) should still use the original Auger transform,
else the number of excited electrons due to the neutralization are 
still normalized to one per incoming ion. 

\printbibliography 
\end{refsection} 